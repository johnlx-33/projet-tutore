\documentclass[a4paper, 11pt]{report}

\usepackage[T1]{fontenc}
\usepackage[utf8]{inputenc}
\usepackage[francais]{babel}
\usepackage{graphicx}
\usepackage[linesnumbered, french]{algorithm2e}
\usepackage{appendix}
\usepackage{algorithm}
\usepackage{algorithmic}
\usepackage{amsfonts}
\usepackage{amsmath}
\usepackage{hyperref}



\title{\: \: \: \: PROJET TUTORÉ\newline Reconstituer le théorème chinois en présence d’erreurs}
\author{Mr LAVOIX John, Mr GRANERO Fabien }
\date{7 mai 2021}

\begin{document}
\maketitle
\newpage
\begin{abstract}

Nous sommes étudiants en troisième année de licence mathématique et informatique. 
Suite au cours d'algèbre des années précédentes 
et au cours d'arithmétique et cryptologie de ce semestre, nous avons décidé d'approfondir nos connaissances dans le domaine des restes chinois. \newline
De plus, voulant tous les 2 poursuivre nos études en nous inscrivant au  master de cryptologie et sécurité informatique, ce projet fut l'occasion de découvrir les prémisses des enseignements qui y sont dispensés.  \newline
\newline
\newline
    Ce rapport est  une synthèse de nos travaux durant ce semestre, sur la reconstitution du théorème des restes chinois en présence d'erreurs.
Grâce au soutien de notre professeur Mr ZEMOR, qui nous a guidé tout au long de ce projet, nous avons découvert et maitrisé plusieurs méthodes pour parvenir à notre objectif. \newline
Pour ce faire, nous avons d'abord pris connaissance du théorème dans son ensemble. Puis dans un second temps, nous avons cherché et appris plusieurs facon de corriger le théorème des restes chinois lors de la présence d'erreurs.

\end{abstract}


\tableofcontents

\newpage

\chapter*{Remerciements}
\: Nous tenons tout d'abord à remercier monsieur CASTAGNOS Guillaume, d'avoir maintenu cette unité d'enseignement malgré la situation sanitaire actuelle.  
Nous remercions par la suite monsieur ZEMOR Gilles de nous avoir formé durant cette période compliquée, d'être resté à notre écoute, et de nous avoir suivi tout au long de ce projet.
\newline 
\newline
\: \: Nous remercions également nos familles et nos proches, pour leur aide et leur soutien moral.

\newpage

\chapter{théorème des restes}
\section{théorème des restes chinois et son histoire}
Les premières apparitions du théorème des restes chinois ont eu lieu au XIII° siècle  dans un livre de mathématique chinois de Qin JUSHIO publié en 1247. Cependant, on avait déjà découvert ce théorème auparavant dans un livre de Sun ZI au III° siècle. Le théorème consiste en:
On pose p1,...,pk des entiers premiers 2 à 2. Pour tout n1,...,nk, il existe un entier x tel que :
\newline
\newline
$ x\equiv n_1 \: (mod \;  p_1)$ 
\newline
$ x\equiv n_2 \: (mod \: p_2)$
\newline
$ x \equiv n_k \:(mod\: p_k)$
\newline
\newline
Il est difficile de trouver des applications immédiates au théorème des restes chinois. 
Pourtant il se révéle trés pratique dans des comptages à grandes échelles pour calculer trés rapidement 
en minimisant les possibilités d'erreurs ;
ainsi qu'en cryptologie où la possibilité de faire des calculs sur de grands nombres avec des valeurs discrétes permet d'assurer une certaine sécurité si toutefois une partie des données fuitait.
Cela s'applique particuliérement dans le domaine des cartes à puces (cartes bancaires, badges, cartes d'abonnement,...), où il y a une interaction avec un terminal pour transmettre un code.
Il est très intéressant de délivrer l'information souhaitée par morceaux, surtout dans le cas ou le terminal ferait l'objet d'une attaque.
Dans ce cas, si l'ensemble des données recupérées par l'attaquant n'est pas trop important, alors il est impossible pour lui de les exploiter.
\newline
\newline
\newline
\newline
Nous allons dans un premier temps démontrer l'unicité de ce théorème.
\newline
Tout d'abord, nous cherchons l'existence d'une solution: \newline
On a $\forall k \in [1;y]  $, $x\equiv n_k (mod \: p_k)$ \newline
$\forall k \in [1;y] $, on note $P_k=\frac{P}{p_k} $ où $P=p_1 \times  ... \times p_y$ \newline
On voit assez simplement que $P_k$ et $p_k$ sont premiers entre eux car tous les entiers $p_i$ sont premiers entre eux 2 à 2. Donc $P_k$ est inversible modulo 
$p_k$, car en faisant le théorème d'euclide, on va trouver $u,v\in \mathbb{Z}$ tel que $P_k\times u + p_k \times v = d$ avec d le PGCD. Or $P_k$ et $p_k$ sont premiers entre eux donc $d=1$ et on trouve facilement l'inverse de $P_k$ modulo $p_k$. \newline
On note alors $u_k$ le nombre tel que $u_k\times P_k + v_k\times p_k= 1$, soit $u_k\times P_k\equiv 1(p_k)$. \newline
Soit $x=\sum_{k = 1}^{y} u_k\times P_k\times n_k$. On pose $i\in [1;y]$ et $k\neq i$ alors $P_i\equiv 0(mode \: p_k)$ car $P_i=p_1\times ... \times p_k \times ...\times p_y$. \newline
On a donc $x\equiv u_i P_i a_i(mod \: p_i)$, mais on sait que $u_i P_i\equiv 1 (mod \: p_i)$ d'où $x\equiv a_i (mod \: p_i)$. Sachant que cette équation est vraie pour tout i, on a x solution du système. \newline
Par conséquent, nous avons prouvé l'existence d'une solution.

Ensuite, nous montrons l'unicité du système. \newline
On a x une solution du système, posons y une autre solution du système. On a alors $x-y\equiv 0( mod p_k)$, soit $p_k$ divise $x-y$. Sachant que les $p_k$ sont premiers 2 à 2 entre eux, on a donc $x-y\equiv 0(mod \: P)$, donc P divise x-y, ou encore que $x\equiv y (mod \: P)$. Par cela, nous avons montré l'unicité modulo P.
\newline
\newline
\newline
\newline
\newline
Pour illustrer ce théorème, nous allons donner un exemple, extrait du livre de Sun ZI qui a proposé une solution.
\newline
Choisissons des objets comme des bonbons: si on les répartit pour 3 enfants, il en reste 2, si on les répartis
pour les 3 enfants et leurs parents (soit 5 personnes), il en reste 3. Enfin si on rajoute les 2 cousins
(soit 7 personnes), il reste alors 2 bonbons. On a donc :
\newline 
\newline
$ x\equiv 2 \: (mod \:  3)$ 
\newline
$ x\equiv 3 \: (mod \: 5)$
\newline
$ x\equiv 2 \:(mod\: 7)$
\newline
\newline 
La question que l'on se pose à présent est : combien y a t'il de bonbons?
\newline
Grâce au théorème des restes chinois, on peut trouver la réponse. Nous avons en réalité plusieurs réponses, c'est-à-dire que tout les nombres congrus à x modulo N sont des bonnes réponses.
Avant de répondre à ce problème, il faut tout d'abord examiner l'algorithme.



\newpage

\section{Notre algorithme}
Notre fonction en Python du théorème des restes chinois étant un peu lourde,
nous allons montrer l'algorithme ci-dessous (voir annexe A).
On obtient donc :
\newline
Soit $p_i$ le i-ème terme de la liste des modulos, on note \newline
$ P_i=\frac{p}{p_i}=p_1 p_2 ... p_{i-1} p_{i+1} ... p_k $   \newline
on a donc $P_i$ et $p_i$  qui sont premiers entre eux. \newline
Il faut alors faire l'algorithme d'Euclide étendu sur $P_i$ et $p_i$, ce qui nous donne: 
$1= P_i u_i + p_i v_i$ 
où 
$u_i, v_i \in  \mathbb{Z}  $
\newline
On pose donc $e_i = u_i P_i$ avec $ e_i \equiv 1 \; (mod \; p_i)$ et $ e_i\equiv 0 \; (mod \; e_j)$ avec $ i\neq j$ \newline
On trouve alors une solution qui est $x=\sum_{i = 1}^{k}{p_i e_i} $.\newline
\newline
\newline
Nous pouvons à présent répondre à l'exemple précédent. Nous avions: \newline
$ x\equiv 2 \: (mod \:  3)$ 
\newline
$ x\equiv 3 \: (mod \: 5)$
\newline
$ x \equiv 2 \:(mod\: 7)$
\newline
On obtient $P_1=5\times 7=35$, $P_2=3\times 7=21 $, et $P_3=3\times 5=15$ \newline
On fait l'algorithme d'Euclide étendu sur $P_1$ et $p_1$ qui donne $-3\times 23 +2\times 35\times 1= 1 $, par conséquent on trouve $e_1=2\times 35$ \newline
Idem sur $P_2$ et $p_2$ qui donne $21\times 1 - 5\times 4=1$, donc $e_2=21$ \newline
Enfin, on a $15\times 1- 7\times 2 = 1 $ avec $e_3=15$ \newline
Le résultat est $x=2\times 35 + 3\times 21 + 2\times 15 =233$.
Nous avons, comme dit précédement, plusieurs résultats qui sont les entiers congrus à 233 modulo 105. \newline
$233\equiv 23 (mod \: 105)$ \newline
Le résultat final est donc $23+105k$, $k \in \mathbb{Z} $.
Si on reprend notre problème, on a donc 23 bonbons.
\newline
\newline
Comme vous pouvez le voir, nous avons utilisé l'algorithme d'Euclide étendu. Celui-ci prend en paramètres a et b, deux entiers.
Il renvoit $d$ le PGCD de $a$ et $b$ et un couple $(u,v) \in \mathbb{Z} $ tel que $d=au+bv$.
\newline
Pour notre utilisation, nous avons un peu modifié cet algorithme, car pour utiliser le théorème des restes chinois, les nombres sont premiers entre eux 2 à 2,
donc $d=1$. De plus, il nous fait gagner une étape car il nous renvoit l'inverse de a modulo b.

\begin{algorithm}
    \caption{algorithme d'euclide étendu}
    \begin{algorithmic}
        \STATE $x_0 \leftarrow 1 $ 
        \STATE $x_1 \leftarrow 0 $
        \STATE $y_0 \leftarrow 0 $ 
        \STATE $y_1 \leftarrow 1 $ 
        \STATE $s \leftarrow 1 $ 
        \STATE $d \leftarrow b $
        \WHILE{$b\neq 0$}
        \STATE $ (q,r) \leftarrow divmod(a,b) $  /* q est le quotient et r le reste de la division euclidienne de a par b */ 
        \STATE $ (a,b) \leftarrow (b,r) $ 
        \STATE $ (x,y) \leftarrow (x_1,y_1)$ 
        \STATE $ (r_1,y_1) \leftarrow (q\times s1 + x_0 , q\times y_1 + y_0) $
        \STATE $ (x_0,y_0) \leftarrow (x,y)$ 
        \STATE $ s \leftarrow -s$ 
        \ENDWHILE
        \RETURN {$s\times x_0 + ((1-s)\div 2)\times d$}
    \end{algorithmic}

\end{algorithm}
\newpage

Maintenant que nous avons présenté les algorithmes principaux du théorème des restes chinois, voici quelques fonctions que nous pourrions qualifier de secondaires au premier abord, mais très utiles dans certains cas. \newline
En premier, nous avons fait un algorithme s'intitulant générateur de cas, qui prend en argument 2 nombres et qui génère les listes des nombres premiers entre eux, et la liste des restes modulo les $p_i$. C'est en
quelque sorte la fonction inverse du théorème des restes chinois. 

\begin{algorithm}
    \caption{algorithme d'encodage}
    \begin{algorithmic}
        \REQUIRE $n>k$
        \STATE $l_n= nombre-facteur-premier(n)$
        \STATE $l_k=[]$
        \FOR {i in range(len(ln))}
        \STATE $l_k=l_k +[k/ln(i)]$
        \ENDFOR
        \RETURN {($l_n$, $l_k$)}
    \end{algorithmic}

\end{algorithm}

Par exemple notre algorithme d'encodage sur $(30030,61)$ renvoie les listes $P=[2,3,5,7,11,13]$ et la liste $N=[1,1,1,5,6,9,10]$ ,
qui représente la liste des facteurs premiers qui composent 30030 et la représentation de 61 modulo les élements de P .  \newline

Nous pouvons remarquer que dans la fonction au dessus, nous utilisons la fonction nombre facteur premier qui prend en paramètre un nombre n et renvoie la liste des ses facteurs premiers. 

\begin{algorithm}
    \caption{nombre facteur premier}
    \begin{algorithmic}
        \STATE $l=[]$
        \STATE $i=2$
        \WHILE {$n\neq 1$} 
        \STATE $c=1$
        \WHILE {$n\%i=0$}
        \STATE $n=n//i$
        \STATE $c=c\times i$
        \ENDWHILE
        \IF {$(n\times c) \%i= 0$} 
        \STATE $l=l+\left[c\right] $
        \ENDIF
        \STATE $i=i+1$
        \ENDWHILE
        \RETURN{$l$}
    \end{algorithmic}
\end{algorithm}

Par exemple nombre facteur premier (210) renvoie la liste $[2,3,5,7]$ car $210=2\times 3\times 5\times 7$

\newpage
\chapter{Presentation du problème}
\section{Le probléme et ses restrictions}
\hbox{Le théorème des restes chinois induit une bijection :}
\begin{center}
  

$[0,N-1] \rightarrow \mathbb{Z}/ p_{1 \mathbb{Z}} \times  ... \times   \mathbb{Z}/ p_{i \mathbb{Z}} $  \newline
$x \rightarrow (x_1,...,x_i) $ 

\end{center}

Oû $ x\equiv x_i \: (mod \: p_i)$
\newline

Malheuresement nous ne pouvons encoder l'ensemble $[0,N-1]$ dans son entièreté et être capable de corriger et faire face aux erreurs.\newline
A  propos du N à choisir ce qui nous intéresse dans le N choisi c'est un N entier produit de plusieurs nombres premiers.\newline 
Car plus il y a de nombres premiers qui composent N plus le systéme des restes chinois engendré par ce N contient d'équations.  \newline
Tout les nombres de 0 à N-1 ne peuvent pas être "encodés" si l'on veut corriger des erreurs.\newline
Il faut determiner une borne B comprise entre 0 et N-1 qui nous servira de "zone" d'encodage.\newline
Pour determiner cette borne on a 2 maniéres :\newline
la premiére est de choisir N et de trouver B en se restreingnant à une sous partie du systéme, le surplus servant de "donnée de parité" ;\newline
la deuxième est de choisir B l'ensemble des entiers que l'ont veut encoder et rajouter des informations donc des équations supplémentaires au systéme engendré par B, donc multiplier B par d'autres nombres premiers.\newline

\newpage

\section{Comment déterminer la borne}


Maintenant comment déterminer B de façon calculatoire ? \newline
Soit $N=p_1\times ...\times p_n$
On pose $x=p_1 \times p_2 \times ... \times p_{n-2t}$ où t est le nombre d'erreurs. \newline
On note la représentation de x par le théorème des restes chinois $m=(0...0 a_1 a_2 ... a_t ... a_{2t})$ \newline
On reçoit le message $r=(0...0 a_1...a_t 0...0)$ \newline
Nous obtenons 2 possibilités d'erreurs, soit: \newline
- $(0...0 a_1...a_t 0...0) $ avec des erreurs sur le t-uplet de $a_1$ à $a_t$, qui correspond à l'encodage de 0. \newline
- $(0...0 a_1...a_t a_{t+1}...a_{2t})$ avec des erreurs sur le t-uplet de $a_{t+1}$ à $a_{2t}$, qui correspond à l'encodage de $x$. \newline
Donc la borne est x pour ne pas laisser de doute au moment de la correction. \newline
\newline
Pour illustrer cette détermination de borne, voici un petit exemple: \newline
Posons $N=2\times 3\times 5\times 7\times 11\times 13 \times 17= 510510$ et $t=2$. \newline
On trouve $x=2\times 3\times 5=30$. Donc pour 2 erreurs, la borne de N est 30.


Dans les algorithmes que vous verrez dans la suite de notre travail, vous pourrez trouver nos calculs de borne. Cependant, ces calculs sont propres à chaque algorithme,
c'est-à-dire que la borne pour un même nombre et un même nombre d'erreurs peut varier en fonction des algorithmes utilisés. \newline
Cela signifie qu'après cette borne, la correction d'erreurs marche parfois, ce n'est donc plus une valeur sûre à cent pour cent, alors qu'avant cette borne, oui.
\newline
La question que nous sommes à présent en droit de nous poser est, est-ce qu'il existe une borne telle que l'on ne puisse corriger aucun nombre supérieur à cette borne? Et la réponse est oui.
Partons d'un n-uplet de départ $x=(x_1, x_2, ..., x_n)$. On pose également, dans le n-uplet reçu, $y$ le résultat possible avec t erreurs. Avec bien sûr, le cardinal des n-uplets reçu qui est infèrieur à la sommes des $p_i$. Dans les n-uplets reçus, il y a un nombre $D$ de résultats possibles. Nous cherchons donc à calculer $D$, qui vaut: \newline
$D=\sum_{I \subset  [n]  \newline |I|=t  }  \prod _{i \in I} p_i - 1 \simeq  \binom{n}{t} p^t  $ avec $[n]$ les entiers de 1 à n et $p=p_n$\newline
Par exemple si $x=(x_1, x_2, x_3, x_4)$ et $t=2$ alors $D=(p_1-1)(p_2-1)+(p_3-1)(p_4-1)+(p_1-1)(p_3-1)+(p_1-1)(p_4-1)+(p_2-1)(p_3-1)+(p_2-1)(p_4-1)$ \newline
On a donc $ |recu| \simeq |B| \times D\leq \sum_{i = 1}^{n} p_i  $ \newline
Donc la borne absolue vaut : $|B| \leq  \frac{\sum_{i= 1}^{n} p_i }{D} $.\newline
Idem que précédemment, faisons un petit exemple. \newline
Posons $N=2\times 3\times 5\times 7\times 11\times 13$ et $t=2$, alors $D\simeq \binom{6}{2}\times 13^2=2535$
$B \leq  \frac{2\times 3\times 5\times 7\times 11\times 13}{D}=\frac{510510}{2535} \simeq 201 $.


\newpage

\chapter{Brute force}

Il existe plusieurs algorithmes de brute force qui permettent de résoudre les erreurs tel que essayer tous les cas possibles. Mais ces algorithmes étant trop lourd à présenter dans ce dossier, nous avons décidé de les inclure en annexes.
Nous avons ainsi choisi, pour la suite de notre travail, d'utiliser des algorithmes compatibles avec notre support.

\section{Notre premier algorithme version naïve }
Tout d'abord, nous avons fait une fonction qui nous montre s'il y a une erreur dans nos listes, car avant de vouloir trouver les erreurs, il faut savoir s'il y en a une (voir annexe 2).
Nous avons donc fait un algorithme qui permet d'enlèver alternativement un élément des listes. Autrement dit, on retire le premier élément des deux listes, puis on ne retire que le deuxième élément, puis que le troisième et ainsi de suite. Nous appliquons alors le théorème des restes chinois à chaque itération. Puis on mets le résultat dans une liste auxiliaire. S'il n'y a pas d'erreur,
lorsqu'on retourne cette liste auxiliaire, on devrait avoir la même valeur partout, en revanche si les valeurs de la liste sont différentes, nous avons une erreur. \newline
\newline
Soit $P=[p_1 , p_2, ... , p_k]$ et $N=[n_1, n_2, ..., n_k]$ \newline
On note $P_i = [p_1,..., p_i-1, p_i+1, p_k]$ et $N_i=[n_1,..., n_i-1, n_i+1, ... , n_k]$ \newline
On note $x_i$ le résultat théorème des restes chinois appliqué à $P_i$ et $N_i$ \newline
Puis on retourne 
$L=[x_1, ..., x_k]$ \newline
Si 
$x_1=x_2=...=x_k$ 
alors il n'y a pas d'erreur. \newline
\newline
Voici un exemple pour bien comprendre cette fonction.\newline
On pose P=[2,3,5,7,11,13] et N=[1, 1, 1, 5, 6, 9] \newline
Si on applique le théorème des restes chinois à ces deux listes, on trouve 510510. Pour vérifier qu'il n'y a pas d'erreur, on applique le théorème à
P1=[3,5,7,11,13] et N1=[1, 1, 5, 6, 9, puis à P2=[2,5,7,11,13] et N2=[1, 1, 5, 6, 9] jusqu'à P6=[2,3,5,7,11,] et N6=[1, 1, 1, 5, 6]\newline
Comme il n'y a pas d'erreur, la fonction va afficher [61, 61, 61, 61, 61, 61] \newline
À présent, nous insérons une erreur dans les listes, on a P=[2,3,5,7,11,13] et N=[1,1,2,5,6,9]. En appliquant la fonction, il sera retourné une liste [6067, 6067, 61, 1777, 607, 1447], où l'on remarque clairement que chaque élément est différent. Donc nous avons au moins une erreur.
Dans cette fonction, on parcourt n fois la liste de n éléments donc nous sommes en $O(n^2)$. De plus, on fait n fois l'algo des restes chinois donc on est en $O(n^2)$. Cette algorithme est donc en $O(n^4)$ avec n la taille de la liste.
Cette fonction a éaglement une borne qui est $ \prod_{i=1}^{n-1} p_i $.
\newline 
\newline
Pour créer notre première fonction de correction, nous nous sommes inspirés de la fonction ci-dessus (voir annexe 3). On s'est dit que si on enlevait 2 éléments des listes, nous pourrions trouver l'erreur. 
En effet, on a vu que si on enlevait 1 élément des listes, le théorème des restes chinois marchait encore, et si on enlève 2 éléments, il marche encore. Donc, en enlevant le premier élément et tous les autres chacun leur
tour, et en appliquantle théorème à chaque fois, on retourne une liste. Puis, on réitére en enlevant le deuxième élément, et les autres chacun leur tour, en re-appliquant les restes chinois à chaque fois, en mettant les résultats dans
une nouvelle liste. Ainsi de suite jusqu'au dernier élément. Cette fonction retourne une liste de liste. Donc la liste où se trouve le même nombre partout, signifie que l'erreur est à la même position.
Faisons plutôt un exemple.
\newline 
Soit $P=[p_1, p_2, ... , p_k]$ et $N=[n_1 , n_2, ... ,n_k]$ \newline
On note $P_ij =[p_1 , ... , p_i-1 , p_i+1 , ... , p_j-1 , p_j+1 , ... , p_k]$ et $N_ij=[n_1, ..., n_i-1, n_i+1, ..., n_j-1, n_j+1, ..., n_k]$ \newline
On note $x_ij$ le résultat du théorème des restes chinois appliqués à $P_ij$ et $N_ij$ \newline
Et on retourne $L=[[x_11,x_12,...x_1k],[x_21,x_22,...,x_2k],...,[x_{1k},x_{2k},...,x_{kk}]]$ \newline
On a retourné une liste de liste. Si la k-ième sous-liste retourne toujours le même nombre alors cela signifie qu'il y a une erreur sur le k-ième élément. \newline`
\newline
Si on récupère l'exemple du paragraphe précédent, où l'on avait découvert une erreur, \newline
on a: $P=[2,3,5,7,11,13]$ et $N=[1,1,2,5,6,9]$ \newline
On applique notre algorithme et il retourne
[[6067, 1062, 61, 1777, 607, 292], [1062, 6067, 61, 347, 607, 677], [61, 61, 61, 61, 61, 61], [1777, 347, 61, 1777, 217, 127], [607, 607, 61, 217, 607, 187], [292, 677, 61, 127, 187, 1447]] \newline
La première liste retournée ne renvoit pas le même résultat donc l'erreur n'est pas sur la première équation. Elle n'est pas non plus sur la deuxième équation.
Cependant, on peut voir sur la troisième liste que le résultat est toujours le même; c'est à dire qu'à chaque fois on a fait le théorème des restes chinois en enlevant la troisième équation et une autre, donc nous avons une erreur sur la toisième équation.
\newline
Voici un exemple d'échec de notre algorithme pour des valeurs prises en dehors de la borne.
$N=[0, 2, 0, 5, 9, 11]$ est la liste représentant 2000 modulo les $p_i$. Posons une erreur sur $N$, tel que $N=[0, 2, 0, 6, 9, 11]$ : l'algorithme renvoie 
[[12725, 2715, 713, 2000, 440, 20], [2715, 7720, 1714, 570, 440, 20], [713, 1714, 3716, 284, 440, 20], [2000, 570, 284, 2000, 50, 20], [440, 440, 440, 50, 440, 20], [20, 20, 20, 20, 20, 20]].
Comme on l'observe, l'algorithme trouve parfois 2 000 mais ne trouve pas l'erreur, ou du moins, il en trouve une mauvaise.
\newline
Cette fonction est très performante comme vous avez pu le voir sur les exemples, cependant il y a quelques défauts. Premièrement, cette fonction ne marche qu'avec une seule erreur,
on peut facilement changer la fonction pour trouver n erreur juste en rajoutant des boucles FOR, mais cela rendrait l'algorithme encore plus lourd et une borne très faible. Deuxièmement, nous avons un problème de borne. Pour trouver la borne,
il suffit de multiplier par les 2 derniers nombres premiers de la liste. \newline
Par exemple pour le nombre 30030, la borne est 30030/(11x13) soit 210, ce qui est très peu.\newline
Enfin, nous avons un problème de complexité soit $O(n^3)$ et à chaque fois on fait le théorème des restes chinois qui est en $O(n^2)$, ce qui donne un algorithme en $O(n^5)$.




\newpage
\section{brute force de hamming}

Notre deuxième algorithme brut force consiste à utilisé la distance de hamming. Soit $x=x_1x_2 ... x_n$ et $y=y_1y_2 ... y_n$, la distance de Hamming c'est
$d_H(x,y)=Card\{  i\vert x_i \neq y_i \}$. Par exemple la distance de Hamming de $x={1,2,3,4}$ et $y={1,2,5,6}$ est 2\newline
Avant de voir cette algorithme, nous devons expliquer la distance minimale. Soit G un groupe n-uplet appartenant à $A^n$. La distance minimale de G est la plus petite distance de hamming sur tout les éléments de G, soit $d(G)=min{\left\{d_h(x,y)\right\} | x\neq y, x,y\in G}$.
D'après Hamming, si la distance minimale est égal à d, alors on peut détecter d-1 erreurs et corriger $\frac{d-1}{2}$ erreurs. \newline


Dans cette algoritme nous utilisons la distance de haming qui prend en entré 2 listes de nombres nommés $A1$ et $A2$, et retoune le distance de Hamming. Du coup, il nous a fallu le coder. \newline
\begin{algorithm}
    \caption{distance de hamming}
    \begin{algorithmic}
        \STATE $n=min(len(A1),len(A2))$
        \STATE $m=max(len(A1),len(A2))$
        \STATE $cpt=m-n$
        \FOR{i in range(n)}
        \IF{$A1[i]\neq A2[i]$}
        \STATE $cpt=cpt+1$
        \ENDIF
        \ENDFOR
        \RETURN{cpt}        
    \end{algorithmic}
\end{algorithm}

Pour ce nouvel algoritme (voir annexe D), nous allons calculer la distance de Hamming entre le uplet à corriger et tout les élements de 0 à la borne. Nous selectionnons tout les éléments de distance de Hamming strictement inférieur à 3 pour pouvoir détecter une erreur. \newline
Pour choisir la borne de cette algorithme pour 1 erreur, on retire les 3 plus grandes équations, c'est à dire si $P=[p_1, ..., p_k]$ alors la borne est $\prod_{i=1} ^{k-3} p_i$. Pour e erreurs, on retire les $3e$ dernières équations. \newline
Faisons un exemple pour illustrer cette algoritme: \newline
Prenons les listes $P=[2,3,5,7,11,13,17]$ et $N=[1,0,0,5,9,10,7]$ pour nos exemples. \newline
Détaillons, pour mieux le comprendre, notre algorithme sur $N=[1,0,0,5,7,10,7]$ c'est à dire que nous mettons une erreur sur le cinquième thermes.
Dans un premier temps, l'algorithme va faire la distance de Hamming entre la liste N et le nombre 0 qui s'écrit $L_O=[0,0,0,0,0,0,0]$ car le reste dans division euclidienne des $p_i$ par 0 est toujours 0.
La distance de hamming de ces 2 listes est 5, car 2 éléments sur les 7 sont identiques entre les 2 listes, donc on ne prete pas attention à ce cas. \newline
On fait la distance de hamming entre le liste N et 1 qui s'écrit $L_1=[1,1,1,1,1,1,1]$ car le reste dans la division euclidienne des $p_i$ par 1 est toujours 1.
La distance de Hamming de ces 2 listes est 6 donc on ne prete pas attention.
On fais la distance de Hamming entre la liste N et 2 qui s'écrit $L_2=[0,2,2,2,2,2,2]$. La distance de Hamming de ces 2 listes est 7 donc on ne prete pas attention. \newline
Il continue ses recherches sur les entiers 3, 4, et ainsi de suite. L'algorithme à découvert une liste qui avait une distance de Hamming infèrieur à 3. Cette liste est la liste $L=[1,0,0,5,9,10,7]$ qui correspond au nombre 75. Donc 75 est un candidat possible à la correspondance de cette uplet.
l'algorithme va continuer ces tests sur tout les nombres jusqu'à notre borne moins 1 qui ici vaut $2\times 3\times 5\times 7 -1=209$. \newline
\newline
De même que pour l'algorithme précédent, voici le même exemple. \newline
Rapelons les faits, $N=[0, 2, 0, 5, 9, 11]$, et on implémente une erreur qui est: $N=[0, 2, 0, 6, 9, 11]$.
L'algorithme retourne [20], c'est une nouvelle fois une erreur.
\newline
La complexité de cette algorithme se calcul assez rapidement. Nous avons une boucle while, on fait donc $b$ fois la distance de Hamming. LA distance de Hamming est donc $O(r)$ où r est la taille du système. On a donc cette algorithme 
qui est en $O(b\times r)$.




\newpage
\chapter{Les fractions continues} 
\section{Définition}
Comme vu précédement la méthode brute force est efficace sur des petits cas de 
correction mais sur de cas plus grand la complexité explose et il est très difficile d'obtenir des résultats en un temps raisonnable.\newline
Il faut donc trouver un moyen plus rapide et optimal. Donc comprenons comment intervient l'erreur sur le n-uplet du systéme d'équation.\newline
Soit m le message envoyé et y le message et l'erreur dans le même message. si l'on fait la différrence de m-y on obtient e l'erreur.
Maintenant cette erreur ce représente par un uplet en supposant par soucis de représentation que les erreurs se trouve au début de l'uplet associé
$e = [e1,e2,e3,...,e_k,0,...,0]$ cette erreur est donc un multiple de tout les $p_i$ nombres premiers pour i>k et k  reste non nul sur les positions erronés.\newline
Donc $y = m + e$ donc si l'on divise y par N notre entier on a $\frac{y}{N}=\frac{m}{N}+\frac{e}{N}$\newline
mais $\frac{e}{N}$ peut se simplifier car hormis les positions érronés 2 dans notre cas e est un multiple de tout les autres $p_i$ qui composent N\newline
Ainsi on a  $\frac{y}{N}=\frac{m}{N}+\frac{e}{N}=\frac{m}{N}+ \frac{e_1 * e_2}{p_1 * p_2 } <\frac{B}{N} $    B étant la borne choisis.\newline
Finalement $\frac{m}{N}=\frac{y}{N} - \frac{e_1 * e_2}{p_1 * p_2 } $ \newline
\newline
Or souvenons-nous que si z est un réel positif et que la fraction rationnelle $\frac{p}{q}$
vérifie $ \left| z - \frac{p}{q} \right| < \frac{1}{2q^2} $. 
Alors $\frac{p}{q}$ est une fraction reduité de z.\newline
\newline

Appliquer à notre probléme chaque réduite de cette fraction liste au dénominateur des candidats potentiels pour les positions érronés.\newline
Lors de l'écriture des fractions continues en pyhton, nous avons du faire appelle à des foncitons segondaires (voir annexe 5). Tout d'abord, il nous a fallu une petite fonction qui calcul la borne maximal pour cette méthode.
\begin{algorithm}
    \caption{borne maximale pour les fractions continues}
    \begin{algorithmic}
        \STATE $borne=1$
        \FOR{i in range(len(N)-1,len(N)-t-1,-1)}
        \STATE $borne=borne\times N[i]$
        \ENDFOR
        \RETURN{borne}
    \end{algorithmic}
\end{algorithm}    
\newpage
Dans un second temps, nous avons utilisé la fonction réduite qui prend en paramètres les 3 entiers 
$k$ , $n$ qui sont le numérateur et le dénominateur de la fraction d'origine et $p$ l'ordre de la réduite de $\frac{k}{n}$ et renvoie $l$ la liste des $a_i$ qui compose la réduite. \newline
\begin{algorithm}
    \caption{fraction reduite}
    \begin{algorithmic}
        \STATE $a=k$
        \STATE $b=n$
        \STATE $q=1$
        \STATE $r=1$
        \STATE $l=[]$
        \STATE $n=0$
        \WHILE{$n<p$}
        \STATE $q=a//b$
        \STATE $r=a\%b$
        \STATE $l=l+[q]$
        \STATE $a=b$
        \STATE $b=r$
        \STATE $n=n+1$
        \IF{$r=0$}
        \RETURN{l}
        \ENDIF
        \ENDWHILE
        \RETURN{l}
    \end{algorithmic}
\end{algorithm}
\newline
Et enfin, nous retrouvons la fonction liste-int-réduite qui prend en paramètre l la liste des $a_i$ qui compose la réduite, 
retourné par la fonction d'avant et qui renvois 2 entiers le numérateur et le dénominateur de la réduite reconstitué
\begin{algorithm}
    \caption{liste entier réduite}
    \begin{algorithmic}
        \STATE $h_2=0$
        \STATE $h_1=1$
        \STATE $k_1=0$
        \STATE $k_2=1$
        \STATE $hp=0$
        \STATE $kp=0$
        \FOR{i in range(len(l))}
        \STATE $hp=l[i]\times h_1+h_2$
        \STATE $kp=l[i]\times k_1+k_2$
        \STATE $h_2=h_1$
        \STATE $h_1=hp$
        \STATE $k_2=k_1$
        \STATE $k_1=kp$
        \ENDFOR
        \RETURN{hp,kp}
    \end{algorithmic}
\end{algorithm}

\newpage
Maintenant parlons de la complexité de cette méthode. 
On parcours la liste des derniers nombres premiers du systéme, pour calculer une borne à ne pas dépasser au denominateur de la fraction cela reduit drastiquement les itérations à faire.
Une fois cette borne determiné on calcule les reduites de  $\frac{y}{n}$ sans que le denominateur qui sert d'indication sur la position des erreurs ne depasse la borne.
A chaque tour de boucle on calcule une reduite qui coute l'ordre de la réduite fois le coût d'une division euclidienne  ,
l'algorithme permettant de calculé une réduite étant une version modifié de l'algorithme d'Euclide arrété à l'ordre de la fraction réduite voulue.
Donc finalement pour calculer toutes les réduites qui mettent en évidence les potentiels position érronés le coût et relativement faible car dépend de la taille du systéme lineairement donc relativement faible et rapide.
Coût que l'on a constater grâce  nos experiences et test. \newline





\newpage
\section{Résumé d'experiences et test}
Au début de nos tests nous avons pratiqué avec des petits cas de 6 à 7 nombre premier et cette méthode ne se révéla pas très efficace. Du aufait que le système étudié est trop petit, la méthode des fractions continues échoue par manque d'équation.
Mais pour tester des cas avec plus d'erreurs nous avons eu besoin de systéme plus grand, nous avons donc directement étendu notre systéme à 51 nombres premiers.
Nous avons pris la borne B = $10^{29}$ ordre de grandeur de la borne théorique pour corriger 15 erreurs.
Dans ce cas et grâce à la méthode des fractions continue nous avons étè en capacité de corriger jusqu'à 7 erreur. \newline
Voici quelque exemple: \newline 
Pour le premier exemple, nous avons choisi de travailler avec les 20 premiers nombre premier, et $[1, 2, 1, 1, 0, 8, 1, 18, 4, 23, 10, 12, 6, 0, 9, 51, 50, 26, 61, 61]$ le n-uplet avec 2 erreurs.
L'algorithme retoune $[[2], [3, 43]]$. On reconstitue le thèorème en enlevant, dans un premier temps l'équation du modulo 3. Et dans un autre temps, nous avons fait le théorème en retirant 
les équations modulo 3 et 43. Nous avons deux valeurs retourné, suite à la reconstitution du thèorème, qui sont 2162561357080315373769686 pour la première et 86842888231 pour le segond. Le bon résultat étant le plus petit, cela 
signifie que l'erreur est sur l'équation modulo 3 et modulo 43. \newline
\newline


De plus, on pose pour les $p_i$ les 50 premiers nombres premiers, et 
$N=[0, 2, 0, 4, 7, 1, 16, 2, 22, 25, 26, 34, 18, 19, 30, 24, 24, 22, 52, 56, 4, 15, 54, 68, 
89,\newline 58, 39, 15, 54, 11, 112, 73, 134, 126, 102, 142, 142, 92, 159, 48, 119, 33, 16, 77, 28,
 \newline 29, 67, 89, 86, 163, 150]$ le n-uplet reçu, c'est à dire avec des erreurs(7). 
Les erreurs sont sur les positions correspondant aux nombres premiers de la liste suivante $[7, 79, 97, 107, 127, 157, 223]$. 
L'algorithme retourne $[[5], [2, 7], [3, 11], \newline [7, 79, 97, 107, 127, 157, 223]]$. Comme nous pouvons l'observer, l'algorithme à trouvé la bonne réponse. Cependant il propose d'autres cas. Par le même biais que l'exemple précédent, nous trouver que la bonne solution est le 7-uplets.


\newpage
 
\chapter*{Conclusion}
Finalement on peut distingué 2 cas. Premiérement si l'on doit corriger quelques erreurs sur un petit système 
les méthodes brute force évoquées précédement sont a préconiser malgrés leur coût trés élevé
et dans un second cas sur des systéme plus grand de taille 20 ou plus avec plus d'erreur alors la méthode des fractions continue
se révéle beaucoup plus efficace voir nécessaire pour ésperé avoir un résultat comparé au méthode brut force.
Dans des cas concret appliqué a la vie de tout les jours on ce retrouve face a des grands cas comme dans le domaine des cartes a puces ect ..
alors la méthode des fraction continue est à choisir .
\newpage

\chapter*{Histoire du théorème}

Comme nous l'avons vu précédemment, les premières trâces connus de ce théorème remonte environ au III° siècle, où un mathématicien nommé maitre SUN exposé un problème dans son "classique". Maître Sun Zi ou Sun Tzu est un mathématicien et astronome chinois, ayant vécu entre le $III^e$ siècle et le $V^e$ siècle.
C'est d'ailleur le besoin d'éditer des calendriers astronomique précis, qui poussa Sun Zi à 
énnoncé le théorème des restes chinois dans son livre  Sun Tzu Suan Ching (" Classique mathématique de Sunzi ") 
pour pouvoir calculé les points de convergence des périodes de revolution des astres
Le problème de Sun ZIest celui énoncé précédemment où le résultat était 23. Ce problème a ensuit disparu prendant plusieurs siècles car on ne retrouve aucune trace, ni de ce problème, ni de de problème ressemblant. Il faut attendre
le XIII° siècle pour retrouver des traces du théorème. Comme par exemple dans le shushu jiuzhang de Qin JIUSHAO, où il sera donné un algorithme général de résolution. C'est une méthode appelé
le dayan qui généralise le procédé de maitre SUN. Le problème de Sunzi apparait également au début du XIII° en Europe, où Leonard de PISE, dans Liber abbaci, écrira pratiquement au mot près, les mêmes solutions
que dans le Sunzi. À partir de son arrivé en Occident, le problème de résolution des congruences simultanées commence à trouver son heure de gloire, où le grand mathématicien Gauss, écrira la définition des congruences et 
leurs constitution en aritméthie moderne.\newline
La question que l'on se pose, est pourquoi était utilisé ce théorème? D'après les écrits et quelque hypothèse, il aurait servie en astronomie, où les chinois, qui adoraient ce domaine, souhaité calculer les correspondances des calendriers.
Par exemple, Qin JIUSHAO cherchait dans combien de temsp, se sera le quatrième jour de l'année, le huitième jour du mois lunaire et le dernier jour du cycle sexagénère. \newline
Aujourd'hui, le théorème est utilisé dans d'autres dommaines tel qu'en arithmétique et en algèbre, notamment sous sa forme générale dans l'arithmétique des corps, 
que se soit au cours de démonstrations théoriques aussi bien que dans des cas pratiques.

Dans le domaine de l'algorithmique, il est par exemple utilisé dans l'algorithme RSA en cryptographie,
et il intervient aussi dans l'algorithme de Silver-Pohlig-Hellman pour le calcul du logarithme discret. 
Il intervient dans l'algorithme de test de primalité de Agrawal et Biswas, développé en 199913.

Il permet de représenter de grands nombres entiers comme n-uplets de restes de divisions euclidiennes.
Sous cette forme, des opérations comme l'addition ou la multiplication peuvent se faire en parallèle en temps constant 
(pas de propagation de retenue). Par contre, la comparaison ou la division ne sont pas triviales 

\chapter*{Annexes}
\begin{appendices}
    Annexe 1
    \begin{algorithm}
        \caption{algorithme du théorème des restes chinois}
        \begin{algorithmic}
            \REQUIRE $len(P)=len(N)$
            \STATE $b=0$
            \FOR {i in range(len(N))}
            \STATE $p_i=1$
            \STATE $pi=P[i]$
            \STATE $k=1$
            \FOR {j in range(len(N))}
            \IF {$j\neq i$}
            \STATE $p_i=p_i\times P[j]$
            \ENDIF
            \ENDFOR
            \WHILE {$k\times p_i \neq pi\times y +1$ }
            \STATE $k=k+1$
            \ENDWHILE
            \STATE $b=b+k\times p_i \times N[i] $
            \ENDFOR
            \RETURN{$b$}
        \end{algorithmic}
    \end{algorithm}
\end{appendices}

\newpage

\begin{appendices}
    Annexe 2
    \begin{algorithm}
        \caption{algoritme pour savoir s'il y a une erreur}
        \begin{algorithmic}
            \REQUIRE $len(P)=len(N)$
            \STATE $L=[]$
            \FOR {i in range(len(N))}
            \STATE $N1=[]$
            \STATE $P1=[]$
            \FOR {j in range(len(N))}
            \IF{$i\neq j$}
            \STATE $N1=N1+[N[j]]$
            \STATE $P1=P1+[P[j]]$
            \ENDIF
            \ENDFOR
            \STATE $l=reste-chinois(N1,P1)$
            \STATE $L=L+[l]$
            \ENDFOR
            \RETURN{$L$}
        \end{algorithmic}
    \end{algorithm}
\end{appendices}

\newpage

\begin{appendices}
    Annexe 3
    \begin{algorithm}
        \caption{brute force correction 1 erreur}
        \begin{algorithmic}
            \REQUIRE $len(P)=len(N)$
            \STATE $L1=[]$
            \FOR {i in range(len(N))}
            \STATE $L=[]$
            \FOR{j in range(len(N))}
            \STATE $N_i=[]$
            \STATE $P_i=[]$
            \FOR{k in range(len(N))}
            \IF{$k\neq i$ and $k\neq j$}
            \STATE $N_i=N_i+[N[k]]$
            \STATE $P_i=P_i+[P[k]]$
            \ENDIF
            \ENDFOR
            \STATE $l=reste-chinois(N_i,P_i)$
            \STATE $L=L+[l]$
            \ENDFOR
            \STATE $L1=L1+[L]$
            \ENDFOR
            \RETURN{$L1$}
        \end{algorithmic}
    \end{algorithm}
\end{appendices}

\newpage

\begin{appendices}
    Annexe 4 \newline
    Cette algorithme prend en paramètre la liste des modulos, la liste des restes, et le nombre d'erreur. Et retourne la liste des candidats correspondant au uplet fourni.
    \begin{algorithm}
        \caption{brute force de hamming}
        \begin{algorithmic}
            \STATE $n=len(N)$
            \STATE $cpt=0 $
            \STATE $N=1$
            \STATE $borne=0$
            \FOR{i in range(n)}
            \STATE $N=N\times P[i]$
            \STATE $borne=borne+ (P[i]-1)$
            \ENDFOR
            \STATE $borne=N//borne$
            \STATE $(a,L-force)= generateur-de-cas(N,cpt) $
            \STATE $l-candidat=[]$
            \WHILE{$cpt \leq borne$}
            \IF{$distance-hamming(L-force,N) \leq nb-erreur$}
            \STATE $l-candidat = l-candidat+[cpt]$
            \ENDIF
            \STATE $cpt=cpt+1$
            \STATE $(a,L-force)=genrateur-de-cas(N,cpt)$
            \ENDWHILE
            \RETURN{l-candidat}
        \end{algorithmic}
    \end{algorithm}
\end{appendices}

\newpage

\begin{appendices}
    Annexe 5
    Cette fonction prend en paramètre la liste des $p_i$ la liste des restes modulos les $p_i$, et le nombre d'erreur et retourne la liste des listes d'équations supposées fausses.
    \begin{algorithm}
        \caption{fraction continue}
        \begin{algorithmic}
            \STATE $k=reste-chinois(P,N)$
            \STATE $p=1$
            \STATE $n=1$
            \STATE $L=[]$
            \FOR{i in range(len(P))}
            \STATE $n=n\times P[i]$
            \ENDFOR
            \STATE $borne=max-borne-frac(N,nb-erreur)$
            \STATE $b=0$
            \WHILE{$b\leq borne$}
            \STATE $L1= fraction-reduite(k,n,p)$
            \STATE $(a,d)=list-int-reduite(L1)$
            \STATE $L2=decompose-liste(b,N)$
            \IF{$L2\neq -1$ and $L2\neq []$}
            \STATE $L=L+[L2]$
            \ENDIF
            \STATE $p=p+1$
            \ENDWHILE
            \RETURN{$L$}
        \end{algorithmic}
    \end{algorithm}
\end{appendices}
\newpage

\chapter*{bibliographie et sitographie}
\url{https://cm2.ens.fr/content/le-problème-des-restes-chinois-questions-sur-ses-origines}
\url{http://www.bibmath.net/dico/index.php?action=affiche&quoi=./c/chinois.html}
\url{http://epsilon.2000.free.fr/Csup/resteschinois.pdf}
\url{http://math.univ-lyon1.fr/irem/IMG/pdf/Th_restes_chinois.pdf}

\newline
\newline
\newline
Kaoru Baba et Kiyosi Yabuuti, Une histoire des mathématiques chinoises, Belin. \newline
Jean-Claude Martzloff, Histoire des mathématiques chinoises, p. 129 \newline
J.-C. Martzloff, Histoire des mathématiques chinoises, p. 296. \newline
Ancienne, écrite par Arnaud Gazagnes et publiée par l’IREM de Reims en 2005 \newline
D. M. Mandelbaum, On a class of arithmetic codes and a decoding algorithm, IEEE Trans. on Information Theory 21(1) 85􏰀88, 1976. \newline
Une révolution de la théorie des nombres GAUSS, RBA \newline

\end{document}