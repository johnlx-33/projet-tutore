\documentclass[a4paper, 11pt]{article}

\usepackage[T1]{fontenc}
\usepackage[utf8]{inputenc}
\usepackage[francais]{babel}
\usepackage{graphicx}

\begin{document}
\title{Reconstituer avec le théorème chinois en présence d’erreurs}
\author{Mr LAVOIX John, Mr GRANERO Fabien }
\date{5 mai 2021}



\begin{abstract}

//passage a reécrire (car copier coller) 
\end{abstract}

\newpage

\tableofcontents

\newpage
\begin{flushleft}
    Nous tenons tout d'abord à remercier Mr CASTAGNOS Guillaume d'avoir était le professeur principale de cette matière et,
en ayant bien expliqué le fonctionnement pour partir sur de bonne base. Nous remercions par la suite Mm ZEMOR Eric pour nous 
avoir suivit, aider et tutoré durant ce semestre.
\newline 
nous remercions également nos familles et nos proches, pour leurs aide et leur soutiens.
\end{flushleft}

\newpage
\section{théorème des restes}
\subsection{théorème des restes chinois et son histoire}
Le théorème des restes chinois viens à la base d’un livre mathématique chinois de Qin JUSHIO publié en 1247. Cependant, on avait déjà découvert ce théorème au part avant dans un livre de Sun ZI au 3° siècle. Le théorème consiste en:
On pose n1,...,nk des entiers premiers 2 à 2. Pour tout a1,...,ak, il existe un entier x tel que :
\newline
$ x\equiv a1 \: (mod \;  n1)$ 
\newline
$ x\equiv a2 \: (mod \: n2)$
\newline
$ xk \equiv ak \:(mod\: nk)$
\newline
Nous pouvons démontrer ce théorème de la façon suivante:
\newline
\newline
Pour illustrer ce théorème, nous allons donner un exemple, mais pas n'importe quelle exemple, celui dont Sun ZI a proposé une solution:
\newline
soient des objets, prenons des bonbons, si on les repartis pour 3 enfants, il en reste 2, si on les répartis
pour les 3 enfants et leurs parents, il en reste (soit 5 personnes), il en reste 3. enfin si on partage ces bonbons avec également les 2 cousins,
(soit 7 personnes) il en reste 2. On a donc :
\newline 
$ x\equiv 2 \: (mod \;  3)$ 
\newline
$ x\equiv 3 \: (mod \: 5)$
\newline
$ xk \equiv 2 \:(mod\: 7$
\newline
La question que l'on se pose à présent est combien y a t'il de bonbons?
\newline
grace au théorème des restes chinois, on peut trouver la réponse.

\subsection{notre algoritme}
Pour faire l'algoritme du théorème des restes chinois, il nous a fallu faire d'autres algoritmes

\newpage
\section{grand 2}

\newpage
\section{grand 3}

\end{document}