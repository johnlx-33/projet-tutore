\documentclass[a4paper, 11pt]{article}

\usepackage[T1]{fontenc}
\usepackage[utf8]{inputenc}
\usepackage[francais]{babel}
\usepackage{graphicx}

\begin{document}
\title {Reconstituer avec le théorème chinois en présence d’erreurs}
\author{Mr LAVOIX John, Mr RANDRIAMALALA Toky, Mr GRANERO Fabien }
\date{5 mai 2021}



\begin{abstract}

//passage a reécrire (car copier coller) 
\end{abstract}

\newpage

\tableofcontents

\newpage
\begin{abstract}
    remerciement
\end{abstract}

\newpage
\section{théorème des restes}
\subsection{théorème des restes chinois et son histoire}
Le théorème des restes chinois viens à la base d’un livre mathématique chinois de Qin JUSHIO publié en 1247. Cependant, on avait déjà découvert ce théorème au part avant dans un livre de Sun ZI au 3° siècle. Le théorème consiste en:
On pose n1,...,nk des entiers premiers 2 à 2. Pour tout a1,...,ak, il existe un entier x tel que :
\newline
$ x\equiv a1 \: (mod \;  n1)$ 
\newline
$ x\equiv a2 \: (mod \: n2)$
\newline
$ xk \equiv ak \:(mod\: nk)$
\newline
Nous pouvons démontrer ce théorème de la façon suivante:

\subsection{notre algoritme}

\newpage
\section{grand 2}

\newpage
\section{grand 3}

\end{document}